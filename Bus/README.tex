\section{要求}
\begin{itemize}
	\item 由于线路数据获取源不同,需要分别讨论不同的交通方式,最终统一将获取的原始数据整理为 GPX 运动文件。
	\item 要求描述线路的坐标串需要有一定的密度,不能仅显示站点的坐标信息,需要将站点之间的路线也描述出来。
\end{itemize}


\section{飞机}
\subsection{数据获取}
\begin{itemize}
	\item 在民航领域,飞机的飞行数据主要是为飞行安全服务的,因此飞行数据检测系统大多都专注与实时检测。
	\item 主要有地面雷达系统和飞机自身的数据收集系统来记录分型数据,其中雷达观测数据一般不对外开放。
	\item 飞机自身使用 GPS 和高度计等设备记录的飞行数据会通过 ADS-B 广播系统实时发送到地面。
	\item 只要飞机经过地面 ADS-B 接收站的覆盖区域,地面就能实时监控飞机飞行数据。
	\item 由于 ADS-B 接受站的成本比雷达站低很多,因此世界各国都在地面建设全覆盖的 ADS-B 接收站。
	\item 另外 ADS-B 数据公开环境也比雷达数据要好得多,有很多网站会汇总航班的 ADS-B 数据并对外提供服务。
	\item 在这些网站上,大多只要输入航班就能获得全程的飞行数据,包括可以用来制作 GPX 文件的位置时间数据。
	\begin{itemize}
		\item 国内的 FlightADSB 可以获取 JSON 和 CSV 格式的飞行数据,限制 1 年内的航班。
		\item 美国的 FlightAware 可以获取 KML 格式的飞行数据,限制 3 个月内,更久的数据需要支付高昂的费用。
		\item 非盈利性的 The OpenSky 提供了 Python API, 可以批量获取数据,还没有试过。
	\end{itemize}
	\item 注册并登录 FlightADSB 账号,找到轨迹查询,输入航班号和日期,在条目中会显示航班简要信息,并提供数据下载渠道。
\end{itemize}


\section{公交地铁}
\subsection{数据获取}
\begin{itemize}
	\item 网上现成的公交地铁线路坐标数据库大多都只含有站点的坐标信息,而没有站点间的线路坐标信息。
	\item 在百度地图和高德地图上查询地铁线路,可以看到站点之间的线路形状,说明地图公司拥有站点间坐标信息。
	\item 经过查看文档,发现在高德地图 API 中,使用``公交信息查询''功能可以获得整条公交地铁线路的坐标串。
	\item 不用百度地图,因为考虑到百度地图提供的坐标是特殊的百度坐标 (BD09),而高德地图提供的坐标仅为火星坐标 (GCJ02).
	\item 高德地图 API 的使用教程在官网上比较详尽,这里就简单说明一下。
	\begin{itemize}
		\item 在高德地图开放平台 (\url{https://lbs.amap.com/}) 注册并认证成为个人开发者。
		\item 进入控制台,创建一个新的应用,选择应用类型为 Web API, 并创建一个 Key.
		\item 查阅 Web API 文档可知,使用服务原理是向高德服务器发送 HTTP GET 请求,获取相关的数据。
		\item 对于 Python 有 \verb|requests| 非标准库包可以发送 HTTP 请求并解析返回结果。
	\end{itemize}
\end{itemize}

\subsection{查询线路 ID}
\begin{itemize}
	\item 高德使用 \verb|id| 唯一确定一条公共交通线路,对向发车的算两条不同的线路。
	\item 因此需要先使用自然语言查询,手动筛选出需要的线路的 ID,再使用该 ID 查询出线路信息,并保存到数据库中。
	\item 高德提供的公交路线关键字查询功能返回的信息量过于庞大,手动筛选只需要少量信息即可,因此要对高德的输出进行筛选。
\end{itemize}

\subsection{使用路径 ID 修复坐标串}


\subsection{修复运动轨迹}
\begin{itemize}
	\item 原始运动轨迹过于密集
	\item 原始运动轨迹过于稀疏
	
\end{itemize}



\section{12306火车}
\begin{itemize}
	\item 
	
\end{itemize}

